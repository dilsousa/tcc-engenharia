% resumo em português
\setlength{\absparsep}{18pt} % ajusta o espaçamento dos parágrafos do resumo
\begin{resumo}
 
  Sistemas que gerenciam residências são cada vez mais comuns nos dias atuais. Este trabalho apresenta uma abordagem teórico-prática de como trazer essas tecnologias que já se fazem presentes no âmbito residencial, para o âmbito condominial. Apresenta possíveis tópicos e áreas que poderiam ser melhoradas através de um sistema simples que poderia gerir além das questões físicas de uma construção, como o acendimento automático de luzes, por exemplo, mas também gerir recursos como o uso de um salão de festas.
  
  Além do apontamento das possibilidades, também é destacado alguns tipos de soluções, que envolvem a parte de software, Hardware, segurança e comunicação.
  
  Como resultado do tema trabalhado e do sistema proposto, é apresentada a montagem de um protótipo que tem por objetivo demonstrar alguns recursos do sistema, exercendo controles básicos de  dois apartamentos em um condomínio, utilizando sistemas de prototipação e Android.

 \textbf{Palavras-chaves}: Automação de condomínio, IOT, Raspberry, Domótica, Android Things.
\end{resumo}

% resumo em inglês
\begin{resumo}[Abstract]
 \begin{otherlanguage*}{english}
   This is the english abstract.

   \vspace{\onelineskip}
 
   \noindent 
   \textbf{Key-words}: key word 1, key word 2, key word 3
 \end{otherlanguage*}
\end{resumo}
