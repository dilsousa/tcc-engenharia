% abtex2-modelo-trabalho-academico.tex, v-1.9.2 laurocesar
%% Copyright 2012-2014 by abnTeX2 group at http://abntex2.googlecode.com/
%%
%% This work may be distributed and/or modified under the
%% conditions of the LaTeX Project Public License, either version 1.3
%% of this license or (at your option) any later version.
%% The latest version of this license is in
%%   http://www.latex-project.org/lppl.txt
%% and version 1.3 or later is part of all distributions of LaTeX
%% version 2005/12/01 or later.
%%
%% This work has the LPPL maintenance status `maintained'.
%%
%% The Current Maintainer of this work is the abnTeX2 team, led
%% by Lauro César Araujo. Further information are available on
%% http://abntex2.googlecode.com/
%%
%% This work consists of the files abntex2-modelo-trabalho-academico.tex,
%% abntex2-modelo-include-comandos and abntex2-modelo-references.bib
%%

% ------------------------------------------------------------------------
% ------------------------------------------------------------------------
% abnTeX2: Modelo de Trabalho Academico (tese de doutorado, dissertacao de
% mestrado e trabalhos monograficos em geral) em conformidade com
% ABNT NBR 14724:2011: Informacao e documentacao - Trabalhos academicos -
% Apresentacao
% ------------------------------------------------------------------------
% ------------------------------------------------------------------------

\documentclass[
	% -- opções da classe memoir --
	12pt,				% tamanho da fonte
	openright,			% capítulos começam em pág ímpar (insere página vazia caso preciso)
	twoside,			% para impressão em verso e anverso. Oposto a oneside
	a4paper,			% tamanho do papel.
	% -- opções da classe abntex2 --
	%chapter=TITLE,		% títulos de capítulos convertidos em letras maiúsculas
	%section=TITLE,		% títulos de seções convertidos em letras maiúsculas
	%subsection=TITLE,	% títulos de subseções convertidos em letras maiúsculas
	%subsubsection=TITLE,% títulos de subsubseções convertidos em letras maiúsculas
	% -- opções do pacote babel --
	english,			% idioma adicional para hifenização
	brazil				% o último idioma é o principal do documento
	]{abntex2}

% ---
% Pacotes básicos
% ---
\usepackage{lmodern}			% Usa a fonte Latin Modern
\usepackage[T1]{fontenc}		% Selecao de codigos de fonte.
\usepackage[utf8]{inputenc}		% Codificacao do documento (conversão automática dos acentos)
\usepackage{lastpage}			% Usado pela Ficha catalográfica
\usepackage{indentfirst}		% Indenta o primeiro parágrafo de cada seção.
\usepackage{color}				% Controle das cores
\usepackage{graphicx}			% Inclusão de gráficos
\usepackage{microtype} 			% para melhorias de justificação
% ---

% ---
% Pacotes adicionais, usados apenas no âmbito do Modelo Canônico do abnteX2
% ---
% Pacotes de citações
% ---
\usepackage[brazilian,hyperpageref]{backref}	 % Paginas com as citações na bibl
\usepackage[alf]{abntex2cite}	% Citações padrão ABNT

% ---
% CONFIGURAÇÕES DE PACOTES
% ---

% ---
% Configurações do pacote backref
% Usado sem a opção hyperpageref de backref
\renewcommand{\backrefpagesname}{Citado na(s) página(s):~}
% Texto padrão antes do número das páginas
\renewcommand{\backref}{}
% Define os textos da citação
\renewcommand*{\backrefalt}[4]{
	\ifcase #1 %
		Nenhuma citação no texto.%
	\or
		Citado na página #2.%
	\else
		Citado #1 vezes nas páginas #2.%
	\fi}%
% ---

% ---
% Informações de dados para CAPA e FOLHA DE ROSTO
% ---
\titulo{Software de alta escalabilidade, performance e disponibilidade aplicada a venda de ingressos online}
\autor{André Formento \and Danilo Hércules Araújo Sousa}
\local{Brasil}
\data{2017, v0.0.1}

\orientador{Nome do professor}
\instituicao{%
  Centro Universitário IBTA
  \par
  Engenharia da computação}
\tipotrabalho{Trabalho de conclusão de curso}
% O preambulo deve conter o tipo do trabalho, o objetivo,
% o nome da instituição e a área de concentração
\preambulo{Monografia apresentada ao Curso de Engenharia da Computação do Centro Universitário IBTA como parte dos requisitos para obtenção de Grau em Engenheiro da Computação.}
% ---


% ---
% Configurações de aparência do PDF final

% alterando o aspecto da cor azul
\definecolor{blue}{RGB}{41,5,195}

% informações do PDF
\makeatletter
\hypersetup{
     	%pagebackref=true,
		pdftitle={\@title},
		pdfauthor={\@author},
    	pdfsubject={\imprimirpreambulo},
	    pdfcreator={LaTeX with abnTeX2},
		pdfkeywords={abnt}{latex}{abntex}{abntex2}{trabalho acadêmico},
		colorlinks=true,       		% false: boxed links; true: colored links
    	linkcolor=blue,          	% color of internal links
    	citecolor=blue,        		% color of links to bibliography
    	filecolor=magenta,      		% color of file links
		urlcolor=blue,
		bookmarksdepth=4
}
\makeatother
% ---

% ---
% Espaçamentos entre linhas e parágrafos
% ---

% O tamanho do parágrafo é dado por:
\setlength{\parindent}{1.3cm}

% Controle do espaçamento entre um parágrafo e outro:
\setlength{\parskip}{0.2cm}  % tente também \onelineskip

% ---
% compila o indice
% ---
\makeindex
% ---

% ----
% Início do documento
% ----
\begin{document}

% Retira espaço extra obsoleto entre as frases.
\frenchspacing

% ----------------------------------------------------------
% ELEMENTOS PRÉ-TEXTUAIS
% ----------------------------------------------------------
% \pretextual

% ---
% Capa
% ---
\imprimircapa
% ---

% ---
% Folha de rosto
% (o * indica que haverá a ficha bibliográfica)
% ---
\imprimirfolhaderosto*
% ---

% ---
% Inserir folha de aprovação
% ---

% Isto é um exemplo de Folha de aprovação, elemento obrigatório da NBR
% 14724/2011 (seção 4.2.1.3). Você pode utilizar este modelo até a aprovação
% do trabalho. Após isso, substitua todo o conteúdo deste arquivo por uma
% imagem da página assinada pela banca com o comando abaixo:
%
% \includepdf{folhadeaprovacao_final.pdf}
%
\begin{folhadeaprovacao}

	\begin{center}
	{\ABNTEXchapterfont\large\imprimirautor}

	\vspace*{\fill}\vspace*{\fill}
	\begin{center}
		\ABNTEXchapterfont\bfseries\Large\imprimirtitulo
	\end{center}
	\vspace*{\fill}

	\hspace{.45\textwidth}
	\begin{minipage}{.5\textwidth}
		\imprimirpreambulo
	\end{minipage}%
	\vspace*{\fill}
	\end{center}

	Trabalho aprovado. \imprimirlocal, 20 de dezembro de 2017:

	\assinatura{\textbf{\imprimirorientador} \\ Orientador}
	\assinatura{\textbf{Professor} \\ Convidado 1}
	\assinatura{\textbf{Professor} \\ Convidado 2}

	\begin{center}
	\vspace*{0.5cm}
	{\large\imprimirlocal}
	\par
	{\large\imprimirdata}
	\vspace*{1cm}
	\end{center}

\end{folhadeaprovacao}
% ---

% Isto é um exemplo de Ficha Catalográfica, ou ``Dados internacionais de
% catalogação-na-publicação''. Você pode utilizar este modelo como referência.
% Porém, provavelmente a biblioteca da sua universidade lhe fornecerá um PDF
% com a ficha catalográfica definitiva após a defesa do trabalho. Quando estiver
% com o documento, salve-o como PDF no diretório do seu projeto e substitua todo
% o conteúdo de implementação deste arquivo pelo comando abaixo:
%
% \begin{fichacatalografica}
%     \includepdf{fig_ficha_catalografica.pdf}
% \end{fichacatalografica}
\begin{fichacatalografica}
	\vspace*{\fill}					% Posição vertical
	\hrule							% Linha horizontal
	\begin{center}					% Minipage Centralizado
	\begin{minipage}[c]{12.5cm}		% Largura

	\imprimirautor

	\hspace{0.5cm} \imprimirtitulo  / \imprimirautor. --
	\imprimirlocal, \imprimirdata-

	\hspace{0.5cm} \imprimirorientadorRotulo~\imprimirorientador\\

	\hspace{0.5cm}
	\parbox[t]{\textwidth}{\imprimirtipotrabalho~--~\imprimirinstituicao,
	\imprimirdata.}\\

	\hspace{0.5cm}
		1. Software
		2. Escalabilidade
		3. Performance
		4. Sistema web.
		5. Microserviço.
		I. Orientador: \imprimirorientador.
		II. \imprimirinstituicao.
		III. \imprimirtitulo.\\

	\hspace{8.75cm} CDU 02:141:005.7\\

	\end{minipage}
	\end{center}
	\hrule
\end{fichacatalografica}
% ---

% isso aqui faz com que o sumário não quebre
% se remover ele gera -2 (menos 2) no número da página
\vspace{\onelineskip}


% resumo em português
\setlength{\absparsep}{18pt} % ajusta o espaçamento dos parágrafos do resumo
\begin{resumo}

  escrever resumo

 \textbf{Palavras-chaves}: Automação de condomínio, IOT, Raspberry, Domótica, Android Things.
\end{resumo}

% resumo em inglês
\begin{resumo}[Abstract]
 \begin{otherlanguage*}{english}
   This is the english abstract.

   \vspace{\onelineskip}

   \noindent
   \textbf{Key-words}: key word 1, key word 2, key word 3
 \end{otherlanguage*}
\end{resumo}


% lista de ilustrações
\pdfbookmark[0]{\listfigurename}{lof}
\listoffigures*
\cleardoublepage

\begin{siglas}
    \item[Persona] É uma visão de uma pessoa sobre o sistema: \citeonline{definicao-de-persona}
    \item[Usuário] Usuário é uma persona e é quem usa o front-end (\autoref{arquitetura-front-end}). Exemplos: dono de um apartamento, porteiro, etc.
\end{siglas}


% sumario
\pdfbookmark[0]{\contentsname}{toc}
\tableofcontents*
\cleardoublepage

% ----------------------------------------------------------
% ELEMENTOS TEXTUAIS
% ----------------------------------------------------------
\textual

% regra de nome de arquivo:
% antes do _ está o nome da parte
% depois do _ está o nome do capítulo
% os nomes não devem ter espaços, caracteres especiais e devem ser separados por hífen


\chapter*[Introdução]{Introdução}
\addcontentsline{toc}{chapter}{Introdução}

Escrever introdução

\chapter{Problema}

Atualmente a venda de ingressos para eventos ocorre em websites.
O comum é que haja muita divulgação na mídia a respeito do evento.
A venda de ingressos costuma ser muito disputada devido a limitação
física que o local onde ocorrerá permite, assim como, a oferta de
lugares privilegiados, ingressos com descontos (estudante, idoso), etc.
Desta forma, muitas pessoas buscam os ingressos logo após o início das
vendas, gerando uma demanda muito alta nas primeiras horas de venda.

De modo geral, a alta demanda em sites que não estão preparados para
ter grande variação de tráfego tendem a ter problemas de performance
e até mesmo sairem temporariamente do ar.

\section{Performance}

\section{Escalabilidade}

\section{Disponibilidade}

\chapter{Objetivos}

\section{Objetivo geral}

\section{Objetivos específicos}


\part{Teoria}

% explicar cada um dos conceitos utilizados

\part{Desenvolvimento do protótipo}

% tecnologias aplicadas


\part{Resultados}

Foi feito o login com sucesso via comunicação com google.
Acendemos a luz. 
Notificamos o usuário...

\chapter*[Conclusão]{Conclusão}
\addcontentsline{toc}{chapter}{Conclusão}

escrever conclusão


% ----------------------------------------------------------
% Referências bibliográficas
% ----------------------------------------------------------
\bibliography{referencias-bibliograficas}

% ----------------------------------------------------------
% Glossário
% ----------------------------------------------------------
%
% Consulte o manual da classe abntex2 para orientações sobre o glossário.
%
%\glossary

\begin{apendicesenv}
    
    % Imprime uma página indicando o início dos apêndices
    \partapendices
    
    \chapter{Algum apêndice}
    
    escreve sobre algum apêndice
    
    \chapter{Outro apêndice}

    escreve sobre outro apêndice
    
\end{apendicesenv}


\begin{anexosenv}
    
    % Imprime uma página indicando o início dos anexos
    \partanexos
    
    \chapter{Algum anexo}

    escreve anexo
    
    \chapter{Outro anexo}
    
    escreve outro anexo
    
\end{anexosenv}


%---------------------------------------------------------------------
% INDICE REMISSIVO
%---------------------------------------------------------------------
%\phantompart
%\printindex
%---------------------------------------------------------------------

\end{document}
